\documentclass[polish,polish,a4paper]{article}
\usepackage[polish]{babel}
\usepackage[T1]{fontenc}
\usepackage[utf8]{inputenc}
\usepackage{pslatex}
\usepackage{pgfplots}
\usepackage{circuitikz} 
\usepackage{setspace}
\usepackage{caption}
\usepackage{amssymb}
\usepackage{amsmath}
%\usetikzlibrary{circuits.ee.IEC}
\usepackage{anysize}
\usepackage{graphicx}
\usepackage{hyperref}
\usepackage{float}
\hypersetup{
	colorlinks=true,
	linkcolor=blue,
	filecolor=magenta,      
	urlcolor=cyan,
}

\marginsize{2.5cm}{2.5cm}{2cm}{2cm}

\newcommand{\PRzFieldDsc}[1]{\sffamily\bfseries\scriptsize #1}
\newcommand{\PRzFieldCnt}[1]{\textit{#1}}
\newcommand{\PRzHeading}[8]{
	%% #1 - nazwa laboratorium
	%% #2 - kierunek 
	%% #3 - specjalność 
	%% #4 - rok studiów 
	%% #5 - symbol grupy lab.
	%% #6 - temat 
	%% #7 - numer lab.
	%% #8 - skład grupy ćwiczeniowej
	
	\begin{center}
		\begin{tabular}{ p{0.32\textwidth} p{0.15\textwidth} p{0.15\textwidth} p{0.12\textwidth} p{0.12\textwidth} }
			
			&   &   &   &   \\
			\hline
			\multicolumn{5}{|c|}{}\\[-1ex]
			\multicolumn{5}{|c|}{{\LARGE #1}}\\
			\multicolumn{5}{|c|}{}\\[-1ex]
			
			\hline
			\multicolumn{1}{|l|}{\PRzFieldDsc{Kierunek}}	& \multicolumn{1}{|l|}{\PRzFieldDsc{Specjalność}}	& \multicolumn{1}{|l|}{\PRzFieldDsc{Rok studiów}}	& \multicolumn{2}{|l|}{\PRzFieldDsc{Symbol grupy lab.}} \\
			\multicolumn{1}{|c|}{\PRzFieldCnt{#2}}		& \multicolumn{1}{|c|}{\PRzFieldCnt{#3}}		& \multicolumn{1}{|c|}{\PRzFieldCnt{#4}}		& \multicolumn{2}{|c|}{\PRzFieldCnt{#5}} \\
			
			\hline
			\multicolumn{4}{|l|}{\PRzFieldDsc{Temat Laboratorium}}		& \multicolumn{1}{|l|}{\PRzFieldDsc{Numer lab.}} \\
			\multicolumn{4}{|c|}{\PRzFieldCnt{#6}}				& \multicolumn{1}{|c|}{\PRzFieldCnt{#7}} \\
			
			\hline
			\multicolumn{5}{|l|}{\PRzFieldDsc{Skład grupy ćwiczeniowej oraz numery indeksów}}\\
			\multicolumn{5}{|c|}{\PRzFieldCnt{#8}}\\
			
			\hline
			\multicolumn{3}{|l|}{\PRzFieldDsc{Uwagi}}	& \multicolumn{2}{|l|}{\PRzFieldDsc{Ocena}} \\
			\multicolumn{3}{|c|}{\PRzFieldCnt{\ }}		& \multicolumn{2}{|c|}{\PRzFieldCnt{\ }} \\
			
			\hline
		\end{tabular}
	\end{center}
}



\begin{document}


	\PRzHeading{Laboratorium Podstaw Elektroniki}{Informatyka}{--}{I}{I3}{Elementy RLC}{3}{Piotr Więtczak(132339), Robert Ciemny(136693), Kamil Basiukajc(136681)}
	\section{Krzywa ładowania pojemności}
	\subsection{Cel zadania}
	Celem tego zadania jest empiryczne wyznaczenie krzywej ładowania pojemności, przy pomocy pomiaru czasu i woltomierza.
	\subsection{Przebieg Ćwiczenia}
	Do przeprowadzenia ćwiczenia użyto rezystorów $1M\Omega$ (rzeczywista wartość $0.986M\Omega$), $1k\Omega$ (rzeczywista wartość $0.972k\Omega$), oraz kondensatora $35V$ $47 \mu F$ (rzeczywista wartość $45.450 \mu F$).
	
	Rozpoznano konfiguracje przełącznika, a następnie przy pomocy prototypowej płytki stykowej zbudowano obwód zaprezentowany poniżej i przeprowadzono pomiary napięć co 10 sekund.
	
	
	%% Obwód 
	\begin{figure}[H]
		\begin{equation*}
		\begin{circuitikz}[american]
		\draw
		(0,4) to [V, l=$10V$, a= $E$] (0,0)
		to (2,0)
		to [european resistor, l=$R$, a=$1k\Omega$] (2,2)
		to [short, -o] (3,2)
		(2.5,1.75) node[anchor,north] {(on)}
		(0,4) to [european resistor, l=$R$, a=$1M\Omega$] (2,4)
		to [short, -o] (3,4)
		(2.5,3.75) node[anchor, north] {on}
		(3,3) node[ocirc] {}
		(2.5,3) node [anchor, west] {off}
		(2,0) to (5,0)
		to [C,l=$C$, a=$47\mu F / 35V$] (5,3)
		to (8,3) 
		to [esource] (8,0)
		to (5,0) 
		(8,1.5)		node [anchor,mid] {V}
		(5,3) to (4.5,3)
		to [short, o-] (3.5,3)
		node[inputarrow, rotate=180] {}	
		(4,3) to (4,4.5)
		(3.75,4.35) to (3.75, 4.5)
		to (4.25,4.5)
		to (4.25,4.35);
		\end{circuitikz}
		\end{equation*}
		\captionof{figure}{Obwód do wyznaczania czasu ładowania pojemności.}
	\end{figure}

	%% Wykres Krzywej ładowania pojemności
		\begin{figure}[H]
			\centering
			\begin{tikzpicture}
			\begin{axis}[
			width=0.9\textwidth,
			height = 0.5\textwidth,
			title={Krzywa ładowania pojemności},
			xlabel={Czas w sekundach},
			ylabel={Wartość napięcia zasilania na kondensatorze w V},
			%
			scaled x ticks = false,
			xmax =130,
			x tick label style={/pgf/number format/fixed},
			xticklabel style = {rotate= 90},
			x label style={at={(axis description cs:0.5,-0,1)},anchor=mid},
			%%
			scaled y ticks = false,
			ymax =10,
			y tick label style={/pgf/number format/fixed},
			y label style={at={(axis description cs:0,0.5)},anchor=mid},
			%%
			legend pos=north west,
			ymajorgrids=true,
			grid style=dashed,
			]
			%%
			\addplot[
			color=black,
			mark=*,
			]
			coordinates {
(0,0.047)(10,1.878)(20,3.433)(30,4.650)(40,5.422)(50,6.174)(60,6.745)(70,7.246)(80,7.608)(90,7.910)(100,8.143)(110,8.336)(120,8.489)
			};
					%%
		\addplot[
		color=blue,
		mark=o,
		]
		coordinates {
(0,0.0)(10,1.916546977223037)(20,3.465778722855799)(30,4.718092926577516)(40,5.7303952301316)(50,6.548685041694435)(60,7.210145766772953)(70,7.744834436531373)(80,8.177047510911702)(90,8.52642491917004)(100,8.808842505857626)(110,9.03713343533714)(120,9.221671335734513)

		};
			\legend{
				$u_{C} (t) $,
				$u_{C} (t) -$  krzywa teoretyczna ,
			}
			\end{axis}
			\end{tikzpicture}
	\end{figure}
	
	\subsection{Wyznaczenie przebiegu prądu ładowania pojemności w czasie, na podstawie bilansu napięć w oczku oraz wartości rezystancji $ R_{1} $}
	\begin{gather*}
	(E(1-e^{\frac{-t}{R_{C}}}))' = \\
	=E'(1-e^{\frac{-t}{R_{C}}}) + E(1' - \ln e^{\frac{-1}{R_{C}}} \cdot e^{\frac{t}{R_{C}}}) =\\
	0 + E(\frac{1}{R_{C}}) e^{\frac{-t}{R_{C}}} = \\
	E\frac{1}{R_{C}} \cdot e^{\frac{-t}{R_{C}}}\\
	I_{C}(t) = \frac{E}{R_{C}}e^\frac{-t}{R_{C}}
	\end{gather*}
	
	\subsubsection*{Wykres wyznaczonej zależności}
	
	
		%%WYKRES zależności UC
	\begin{figure}[H]
		\centering
		\begin{tikzpicture}
		\begin{axis}[
		width=0.9\textwidth,
		height = 0.5\textwidth,
		title={Wykres zależności $i_{C}(t)$},
		xlabel={Czas w sekundach},
		ylabel={Prąd ładowania w A},
		%
		scaled x ticks = false,
		x tick label style={/pgf/number format/fixed},
		xticklabel style = {rotate= 90},
		x label style={at={(axis description cs:0.5,-0,1)},anchor=mid},
		%%
		scaled y ticks = false,
		y tick label style={/pgf/number format/fixed},
		y label style={at={(axis description cs:0,0.5)},anchor=mid},
		%%
		legend pos=north east,
		ymajorgrids=true,
		grid style=dashed,
		]
		%%
		\addplot[
		color=black,
		mark=*,
		]
		coordinates {
(10,0.025927893535392457)(20,0.00859428567998197)(30,0.0033934946110907947)(40,0.0011531808608643104)(50,0.0004923900754508685)(60,0.00020312387799445862)(70,0.0000879830086544996)(80,0.00003564640204661098)(90,0.000014468376558386956)(100,0.0000056999116870231325)(110,0.0000022294051805219755)(120,0.0000008550321692950958)

		};
		%%
		\legend{
			$i_{C}(t)$,
		}
		\end{axis}
		\end{tikzpicture}
	\end{figure}
	
	
\section{Obwód RC zasilany prądem przemiennym}
	
\subsection{Cel zadania}
	Obserwacja zmiany skutecznej wartości prądu w obwodzie w funkcji częstotliwości pobudzenia
	
	%%WYKRES zależności UC
	\begin{figure}[H]
		\centering
		\begin{tikzpicture}
		\begin{axis}[
		width=0.9\textwidth,
		height = 0.5\textwidth,
		title={Wykres zależności \underline{$U_{C}$} $= f(\omega)$},
		xlabel={Częstotliwość w kHz},
		ylabel={Wartości skuteczne napięć na pojemności w V},
		%
		scaled x ticks = false,
		x tick label style={/pgf/number format/fixed},
		xticklabel style = {rotate= 90},
		x label style={at={(axis description cs:0.5,-0,1)},anchor=mid},
		%%
		scaled y ticks = false,
		y tick label style={/pgf/number format/fixed},
		y label style={at={(axis description cs:0,0.5)},anchor=mid},
		%%
		legend pos=south west,
		ymajorgrids=true,
		grid style=dashed,
		]
		%%
		\addplot[
		color=black,
		mark=*,
		]
		coordinates {
(1,1.690)(2,1.730)(4,1.730)(6,1.730)(8,1.730)(10,1.700)(12,1.680)(14,1.660)(16,1.720)(18,1.650)(20,1.710)
		};
		%%
		\legend{
			\underline{$U_{C}$} $= f(\omega)$,
		}
		\end{axis}
		\end{tikzpicture}
	\end{figure}
	
	%% WYKRES zależności IC
	\begin{figure}[H]
	\centering
	\begin{tikzpicture}
	\begin{axis}[
	width=0.9\textwidth,
	height = 0.5\textwidth,
	title={Wykres zależności $\underline{I_{C}} = \frac{\underline{U_{R}}}{R}$},
	xlabel={Częstotliwość w kHz},
	ylabel={Wartości skuteczne prądu w obwodzie w $ mA$},
	%
	scaled x ticks = false,
	x tick label style={/pgf/number format/fixed},
	xticklabel style = {rotate= 90},
	x label style={at={(axis description cs:0.5,-0,1)},anchor=mid},
	%%
%	scaled y ticks = false,
%	y tick label style={/pgf/number format/fixed},
	y label style={at={(axis description cs:0,0.5)},anchor=mid},
	%%
	legend pos=north west,
	ymajorgrids=true,
	grid style=dashed,
	]
	%%
	\addplot[
	color=black,
	mark=*,
	]
	coordinates {
(1,0.107)(2,0.209)(4,0.41)(6,0.573)(8,0.744)(10,0.875)(12,0.985)(14,1.08)(16,1.15)(18,1.22)(20,1.28)
	};
	%%
	\legend{
		\underline{$I_{C}$},
	}
	\end{axis}
	\end{tikzpicture}
\end{figure}


\subsection{Wybrana częstotliwość pobudzenia}
	Wybrano częstotliwość 20kHz.
	
	Wartości skuteczne napięć:
	\begin{itemize}
		\item na źródle: $ 1.71V  $
		\item na rezystorze: $1.28V$
	\end{itemize}

	$\triangle x = 4.80 \mu s$
	
\section{Układ RL}

	%%Wykres zależności UL
	\begin{figure}[H]
	\centering
	\begin{tikzpicture}
	\begin{axis}[
	width=0.9\textwidth,
	height = 0.5\textwidth,
	title={Wykres zależności \underline{$U_{L}$} $= f(\omega)$},
	xlabel={Częstotliwość w kHz},
	ylabel={Wartości skuteczne napięć na pojemności w V},
	%
	scaled x ticks = false,
	x tick label style={/pgf/number format/fixed},
	xticklabel style = {rotate= 90},
	x label style={at={(axis description cs:0.5,-0,1)},anchor=mid},
	%%
	%	scaled y ticks = false,
	%	y tick label style={/pgf/number format/fixed},
	y label style={at={(axis description cs:0,0.5)},anchor=mid},
	%%
	legend pos=north west,
	ymajorgrids=true,
	grid style=dashed,
	]
	%%
	\addplot[
	color=black,
	mark=*,
	]
	coordinates {
(1,1.730)(2,1.760)(4,1.750)(6,1.790)(8,1.810)(10,1.780)(12,1.780)(14,1.790)(16,1.820)(18,1.830)(20,1.810)
	};
	%%
	\legend{
		\underline{$U_{L}$},
	}
	\end{axis}
	\end{tikzpicture}
\end{figure}

	%%%Wykres załeżności IL
	\begin{figure}[H]
	\centering
	\begin{tikzpicture}
	\begin{axis}[
	width=0.9\textwidth,
	height = 0.5\textwidth,
title={Wykres zależności $\underline{I_{L}} = \frac{\underline{U_{R}}}{R}$},
	xlabel={Częstotliwość w kHz},
	ylabel={Wartości skuteczne prądu w obwodzie w $ mA $},
	%
	scaled x ticks = false,
	x tick label style={/pgf/number format/fixed},
	xticklabel style = {rotate= 90},
	x label style={at={(axis description cs:0.5,-0,1)},anchor=mid},
	%%
	%	scaled y ticks = false,
	%	y tick label style={/pgf/number format/fixed},
	y label style={at={(axis description cs:0,0.5)},anchor=mid},
	%%
	legend pos=north east,
	ymajorgrids=true,
	grid style=dashed,
	]
	%%
	\addplot[
	color=black,
	mark=*,
	]
	coordinates {
(1,1.600)(2,1.560)(4,1.350)(6,1.150)(8,0.947)(10,0.810)(12,0.718)(14,0.632)(16,0.567)(18,0.501)(20,0.465)
	};
	%%
	\legend{
		\underline{$\underline{I_{L}} $},
	}
 	\end{axis}
	\end{tikzpicture}
\end{figure}

\subsection{Wybrana częstotliwość pobudzenia}
Wybrano częstotliwość 20kHz.

Wartości skuteczne napięć:
\begin{itemize}
	\item na źródle: $ 1.85V  $
	\item na rezystorze: $447m$
\end{itemize}

$\triangle x = 11.2 \mu s$

\end{document}


