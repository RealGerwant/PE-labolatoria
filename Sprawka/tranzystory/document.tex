\documentclass[polish,polish,a4paper]{article}
\usepackage[polish]{babel}
\usepackage[T1]{fontenc}
\usepackage[utf8]{inputenc}
\usepackage{pslatex}
\usepackage{pgfplots}
\usepackage{circuitikz} 
\usepackage{setspace}
\usepackage{caption}
\usepackage{amssymb}
\usepackage{amsmath}
%\usetikzlibrary{circuits.ee.IEC}
\usepackage{anysize}
\usepackage{graphicx}
\usepackage{hyperref}
\usepackage{float}
\hypersetup{
	colorlinks=true,
	linkcolor=blue,
	filecolor=magenta,      
	urlcolor=cyan,
}

\marginsize{2.5cm}{2.5cm}{2cm}{2cm}

\newcommand{\PRzFieldDsc}[1]{\sffamily\bfseries\scriptsize #1}
\newcommand{\PRzFieldCnt}[1]{\textit{#1}}
\newcommand{\PRzHeading}[8]{
	%% #1 - nazwa laboratorium
	%% #2 - kierunek 
	%% #3 - specjalność 
	%% #4 - rok studiów 
	%% #5 - symbol grupy lab.
	%% #6 - temat 
	%% #7 - numer lab.
	%% #8 - skład grupy ćwiczeniowej
	
	\begin{center}
		\begin{tabular}{ p{0.32\textwidth} p{0.15\textwidth} p{0.15\textwidth} p{0.12\textwidth} p{0.12\textwidth} }
			
			&   &   &   &   \\
			\hline
			\multicolumn{5}{|c|}{}\\[-1ex]
			\multicolumn{5}{|c|}{{\LARGE #1}}\\
			\multicolumn{5}{|c|}{}\\[-1ex]
			
			\hline
			\multicolumn{1}{|l|}{\PRzFieldDsc{Kierunek}}	& \multicolumn{1}{|l|}{\PRzFieldDsc{Specjalność}}	& \multicolumn{1}{|l|}{\PRzFieldDsc{Rok studiów}}	& \multicolumn{2}{|l|}{\PRzFieldDsc{Symbol grupy lab.}} \\
			\multicolumn{1}{|c|}{\PRzFieldCnt{#2}}		& \multicolumn{1}{|c|}{\PRzFieldCnt{#3}}		& \multicolumn{1}{|c|}{\PRzFieldCnt{#4}}		& \multicolumn{2}{|c|}{\PRzFieldCnt{#5}} \\
			
			\hline
			\multicolumn{4}{|l|}{\PRzFieldDsc{Temat Laboratorium}}		& \multicolumn{1}{|l|}{\PRzFieldDsc{Numer lab.}} \\
			\multicolumn{4}{|c|}{\PRzFieldCnt{#6}}				& \multicolumn{1}{|c|}{\PRzFieldCnt{#7}} \\
			
			\hline
			\multicolumn{5}{|l|}{\PRzFieldDsc{Skład grupy ćwiczeniowej oraz numery indeksów}}\\
			\multicolumn{5}{|c|}{\PRzFieldCnt{#8}}\\
			
			\hline
			\multicolumn{3}{|l|}{\PRzFieldDsc{Uwagi}}	& \multicolumn{2}{|l|}{\PRzFieldDsc{Ocena}} \\
			\multicolumn{3}{|c|}{\PRzFieldCnt{\ }}		& \multicolumn{2}{|c|}{\PRzFieldCnt{\ }} \\
			
			\hline
		\end{tabular}
	\end{center}
}



\begin{document}


	\PRzHeading{Laboratorium Podstaw Elektroniki}{Informatyka}{--}{I}{I3}{Tranzystory}{5}{Piotr Więtczak(132339), Robert Ciemny(136693), Kamil Basiukajc(136681)}
\section{Charakterystyka bramkowa nMOS}

\subsection{Cel zadania}

Wyznaczenie empirycznej zależności pomiędzy sygnałem sterującym a sterowanym.

\subsection{Przebieg zadania}

Przy pomocy poniższego układu dokonano serii pomiarów wartości prądu drenu $I_{D}$ w zakresie $<0..5> V$. Wyniki pomiarów przedstawiono w poniższej tabeli.


%%  rys 1.6.
\begin{figure}[H]
	\begin{equation*}
	\begin{circuitikz}[american]
	\draw
	(0,0) to (6,0)
	(0,0) to[V,v=$5V$] (0,3)
	to (0,5)
	to [esource](3,5)
	to [european resistor, l=$R_{1}$, a=$1k\Omega$](6,5)
	(6,3) node[nmos] (mos) {}
	(mos.drain) node[anchor = west] {D}
	(mos.source) node[anchor = west] {S}
	(mos.gate) node[anchor = south] {G}
	(6,0) to (mos.source)
	(mos.drain) to (6,5)
	(mos.gate) to (3,3)
	(3,0) to [V,v=$0..5V$] (3,3)
	(4.5,0) to [esource] (4.5,3)
	(4.5,1.45) node[anchor = mid] {V}
	(1.5,4.95) node[anchor = mid] {mA}
	(3,0) node[ground] {};
	\end{circuitikz}
	\end{equation*}
	\captionof{figure}{Układ do badania charakterystyki bramkowej tranzystora nMOS}
\end{figure}

%tabelka	
	\begin{figure}[H]
\begin{spacing}{1.5}
		\begin{equation*}
\begin{array}{|r|r|}
\hline
\multicolumn{1}{|c|}{ $Napięcie$}&\multicolumn{1}{|c|}{I_{D}}\\
\multicolumn{1}{|c|}{ U_{GS}$ $[V]}&\multicolumn{1}{|c|}{[mA]}\\\hline
0.0&0.000\\
0.5&0.000\\
1.0&0.000\\
1.5&0.000\\
2.0&1.360\\
2.1&3.641\\
2.2&4.905\\
2.3&4.699\\
2.4&4.987\\
2.5&4.986\\
3.0&5.000\\
3.5&5.009\\
4.0&5.011\\
4.5&5.014\\
5.0&5.015\\\hline
\end{array}
\end{equation*}
\end{spacing}
\captionof{table}{Tabela przedstawiająca wyniki pomiarów}
	\end{figure}

	%%WYKRES 
\begin{figure}[H]
	
	\centering
	\begin{tikzpicture}
	\begin{axis}[
	width=0.9\textwidth,
	height = 0.5\textwidth,
	title={Wykres przedstawiający wyniki pomiarów},
	ylabel={wartości prądu drenu $I_{d}$ $[ms]$},
	xlabel ={wartości napięcia }
	%
	scaled x ticks = false,
	xtick distance = 0.5,
	x tick label style={/pgf/number format/fixed},
	xticklabel style = {rotate= 90},
	x label style={at={(axis description cs:0.5,-0.15)},anchor=north},
	%%
	ytick distance = 1,
	scaled y ticks = false,
	y tick label style={/pgf/number format/fixed},
	y label style={at={(axis description cs:-0.05,0.85)},anchor=east},
	%%
	legend pos=north west,
	ymajorgrids=true,
	grid style=dashed,
	]
	%%
	\addplot[
	color=black,
	mark=*,
	]
	coordinates {
(0.0,0.000)(0.5,0.000)(1.0,0.000)(1.5,0.000)(2.0,1.360)(2.1,3.641)(2.2,4.905)(2.3,4.699)(2.4,4.987)(2.5,4.986)(3.0,5.000)(3.5,5.009)(4.0,5.011)(4.5,5.014)(5.0,5.015)

	};
	%%
	\addplot[
	color=blue,
	mark=triangle,
	]
	coordinates {
		(0,0)
		
	};
	\legend{
		$I_{D} -pomiary$,
	}
	%%
	\end{axis}
	\end{tikzpicture}
\end{figure}

\section{Charakterystyka bramkowa pMOS}

%%  rys 1.7.
\begin{figure}[H]
	\begin{equation*}
	\begin{circuitikz}[american]
	\draw
	(0,0) to (4.5,0)
	(4.5,0) to [esource] (8,0)
	(0,0) to[V,v=$5V$] (0,3)
	to (0,5)
	to (8,5)
	(8,3) node[pmos] (mos) {}
	(mos.drain) node[anchor = west] {D}
	(mos.source) node[anchor = west] {S}
	(mos.gate) node[anchor = south] {G}
	(8,0) to [european resistor, l=$R_{7}$, a=$1k\Omega$] (mos.drain)
	(mos.source) to (8,5)
	(mos.gate) to (3,3)
	(3,0) to [V,v=$0..5V$] (3,3)
	(4.5,0) to [esource] (4.5,3)
	(4.5,1.45) node[anchor = mid] {V}
	(3,0) node[ground] {}
	(6.25,0) node[anchor = mid] {mA};
	\end{circuitikz}
	\end{equation*}
	\captionof{figure}{Układ do badania charakterystyki bramkowej tranzystora pMOS}
\end{figure}

%tabelka
	\begin{figure}[H]
	\begin{spacing}{1.5}
		\begin{equation*}
		\begin{array}{|r|r|}
		\hline
		\multicolumn{1}{|c|}{ $Napięcie$}&\multicolumn{1}{|c|}{I_{D}}\\
		\multicolumn{1}{|c|}{ $Bramka-$ U_{GS}$ $[V]}&\multicolumn{1}{|c|}{[mA]}\\\hline
0.0&5.000\\
0.5&5.000\\
1.0&5.000\\
1.5&5.000\\
2.0&4.972\\
2.1&4.950\\
2.2&4.902\\
2.3&4.721\\
2.4&4.166\\
2.5&2.793\\
2.6&1.050\\
2.7&0.563\\
2.8&0.249\\
2.9&0.044\\
3.0&0.023\\
3.5&0.000\\
4.0&0.000\\
4.5&0.000\\
5.0&0.000\\\hline
		\end{array}
		\end{equation*}
	\end{spacing}
\end{figure}

	%%WYKRES 
\begin{figure}[H]
	
	\centering
	\begin{tikzpicture}
	\begin{axis}[
	width=0.9\textwidth,
	height = 0.5\textwidth,
	title={Czasy obliczania etykiet dla grafów},
	xlabel={Liczba wierzchołków},
	ylabel={Czas obliczania w misekundach},
	%
	scaled x ticks = false,
	xtick distance = 0.5,
	x tick label style={/pgf/number format/fixed},
	xticklabel style = {rotate= 90},
	x label style={at={(axis description cs:0.5,-0.15)},anchor=north},
	%%
	ytick distance = 1,
	scaled y ticks = false,
	y tick label style={/pgf/number format/fixed},
	y label style={at={(axis description cs:-0.05,0.85)},anchor=east},
	%%
	legend pos=north west,
	ymajorgrids=true,
	grid style=dashed,
	]
	%%
	\addplot[
	color=black,
	mark=*,
	]
	coordinates {
(0.0,5.000)(0.5,5.000)(1.0,5.000)(1.5,5.000)(2.0,4.972)(2.1,4.950)(2.2,4.902)(2.3,4.721)(2.4,4.166)(2.5,2.793)(2.6,1.050)(2.7,0.563)(2.8,0.249)(2.9,0.044)(3.0,0.023)(3.5,0.000)(4.0,0.000)(4.5,0.000)(5.0,0.000)
		
	};
	%%
	\addplot[
	color=blue,
	mark=triangle,
	]
	coordinates {
		(0,0)
		
	};
	\legend{
		$I_{D} -pomiary$,
	}
	%%
	\end{axis}
	\end{tikzpicture}
\end{figure}

\section{Charakterystyka drenowa nMOS}


%%  rys 1.8.
\begin{figure}[H]
	\begin{equation*}
	\begin{circuitikz}[american]
	\draw
	(0,0) to (6,0)
	(0,0) to[V,v=$0..10V$] (0,3)
	to (0,5)
	to [esource](3,5)
	to [european resistor, l=$R_{2}$, a=$1k\Omega$](6,5)
	(6,3) node[nmos] (mos) {}
	(mos.drain) node[anchor = west] {D}
	(mos.source) node[anchor = west] {S}
	(mos.gate) node[anchor = south] {G}
	(6,0) to (mos.source)
	(mos.drain) to (6,5)
	(mos.gate) to (3,3)
	(3,0) to [V,v=$5V$] (3,3)
	(1.5,4.95) node[anchor = mid] {mA}
	(3,0) node[ground] {}
	(6,0) to (7.5,0)
	to [esource] (7.5,5)
	to(6,5)
	(7.5,2.45) node[anchor = mid] {V};
	\end{circuitikz}
	\end{equation*}
	\captionof{figure}{Układ do badania charakterystyki drenowej tranzystora nMOS}
\end{figure}

%tabelka	
\begin{figure}[H]
	\begin{spacing}{1.5}
		\begin{equation*}
		\begin{array}{|r|r|}
		\hline
		\multicolumn{1}{|c|}{ $Napięcie$}&\multicolumn{1}{|c|}{I_{D}}\\
		\multicolumn{1}{|c|}{ $Bramka-$ U_{GS}$ $[V]}&\multicolumn{1}{|c|}{[mA]}\\\hline
0&0.000\\
1&1.097\\
2&2.172\\
3&3.140\\
4&4.125\\
5&5.114\\
6&5.983\\
7&6.950\\
8&8.160\\
9&9.216\\
10&10.158\\
\hline
		\end{array}
		\end{equation*}
	\end{spacing}
\end{figure}


%%  rys 1.9.
\begin{figure}[H]
	\begin{equation*}
	\begin{circuitikz}[american]
	\draw
	(0,0) to (6,0)
	(0,0) to[V,v=$0..10V$] (0,3)
	to (0,5)
	to [esource](3,5)
	to [european resistor, l=$R_{3}$, a=$1k\Omega$](6,5)
	(6,3) node[nmos] (mos) {}
	(mos.drain) node[anchor = west] {D}
	(mos.source) node[anchor = west] {S}
	(mos.gate) node[anchor = south] {G}
	(6,0) to (mos.source)
	(mos.drain) to (6,5)
	(mos.gate) to (5,3)
	(3,3) to [european resistor, l=$R_{4}$, a= $1k\Omega$] (5,3)
	(5,0) to [european resistor, l=$R_{5}$ , a=$1k\Omega$] (5,3)
	(3,0) to [V,v=$5V$] (3,3)
	(1.5,4.95) node[anchor = mid] {mA}
	(3,0) node[ground] {}
	(6,0) to (7.5,0)
	to [esource] (7.5,5)
	to(6,5)
	(7.5,2.45) node[anchor = mid] {V};
	\end{circuitikz}
	\end{equation*}
	\captionof{figure}{Układ do badania charakterystyki drenowej dla obniżonego napięcia bramki}
\end{figure}




\section{Charakterystyka drenowa pMOS}


%%  rys 1.10.
\begin{figure}[H]
	\begin{equation*}
	\begin{circuitikz}[american]
	\draw
	(0,0) to (3,0)
	to [esource] (5,0)
	(4,0) node[anchor = mid] {mA}
	(0,0) to[V,v=$0..10V$] (0,3)
	to (0,5)
	to (5,5)
	(5,3) node[pmos] (mos) {}
	(mos.drain) node[anchor = west] {D}
	(mos.source) node[anchor = west] {S}
	(mos.gate) node[anchor = south] {G}
	(5,0) to [european resistor, l=$R_{6}$, a=$1k\Omega$] (mos.drain)
	(mos.source) to (5,5)
	(mos.gate) to (3,3)
	(3,0) to [V,v=$5V$] (3,3)
	(3,0) node[ground] {}
	(5,0) to (6.5,0)
	to [esource] (6.5,5)
	to (5,5)
	(6.5,2.45) node[anchor = mid] {V};
	\end{circuitikz}
	\end{equation*}
	\captionof{figure}{Układ do badania charakterystyki drenowej tranzystora pMOS}
\end{figure}




\section{Tranzystor nMOS jako przełącznik}




%%  rys 1.11.
\begin{figure}[H]
	\begin{equation*}
	\begin{circuitikz}[american]
	\draw
	(0,0) to (5,0)
	to [esource] (7,0)
	(6,0) node[anchor = mid] {mA}
	(0,0) to[V,v=$0..10V$] (0,3)
	to (0,5)
	to (7,5)
	(7,3) node[pmos] (mos) {}
	(mos.drain) node[anchor = west] {D}
	(mos.source) node[anchor = west] {S}
	(mos.gate) node[anchor = south] {G}
	(7,0) to [european resistor, l =$R_{8}$ , a=$1k\Omega$](mos.drain)
	(mos.source) to (7,5)
	(mos.gate) to (5,3)
	(3,3) to [european resistor, l = $R_{9}$, a = $1k\Omega$] (5,3)
	(3,0) to [V,v=$5V$] (3,3)
	(5,0) to [european resistor, l = $R_{10}$, a = $1k\Omega$] (5,3)
	(3,0) node[ground] {}
	(7,0) to (8.5,0)
	to [esource] (8.5,5)
	to (7,5)
	(8.5,2.45) node[anchor = mid] {V};
	\end{circuitikz}
	\end{equation*}
	\captionof{figure}{Układ do badania charakterystyki drenowej dla obniżonego napięcia bramki pMOS}
\end{figure}

%%  rys 1.14.
\begin{figure}[H]
	\begin{equation*}
	\begin{circuitikz}[american]
	\draw
	(0,0) to (6,0)
	to [V,v=$10V$] (6,3)
	(6,3) to  [european resistor, a = $R_{11}$, l= $1k\Omega$](4,3)
	(2,3) to [leD-,invert,a=LED1] (4,3)
	(2,2) node[nmos] (mos) {}
	(mos.drain) node[anchor = east] {D}
	(mos.source) node[anchor = east] {S}
	(mos.gate) node[anchor = south] {G}
	(2,3) to (mos.drain)
	(2,0) to (mos.source)
	(mos.gate) to (0,2)
	(0,2) to [european resistor, a =$R_{12}$, l=$1M\Omega$] (0,0)
	(0,2) to [short, -*] (-1,2)
	(6,3) to (6,4.5)
	to (-1,4.5)
	to [short,-*] (-1,3)
	(-1,2.5) node[anchor = mid] {A-A};
	\end{circuitikz}
	\end{equation*}
	\captionof{figure}{Schemat układu do badania tranzystora nMOS w roli przełącznika}
\end{figure}

%%  rys 1.15.
\begin{figure}[H]
	\begin{equation*}
	\begin{circuitikz}[american]
	\draw
	(0,0) to (6,0)
	to [V,v=$10V$] (6,3)
	(6,3) to  [european resistor, a = $R_{13}$, l= $1k\Omega$](4,3)
	(2,3) to [leD-,invert , a =LED2] (4,3)
	(2,2) node[nmos] (mos) {}
	(mos.drain) node[anchor = east] {D}
	(mos.source) node[anchor = east] {S}
	(mos.gate) node[anchor = south] {G}
	(2,3) to (mos.drain)
	(2,0) to (mos.source)
	(mos.gate) to (0,2)
	(0,2) to [european resistor, a =$R_{14}$, l=$47k\Omega$] (0,0)
	(0,0) to (-3,0)
	to [C,l=$C1$,a=$100 \mu F$] (-3,2)
	to (0,2)
	(-3,2) to [short, -*] (-4,2)
	(6,3) to (6,4.5)
	to (-4,4.5)
	to [short,-*](-4,3)
	(-4,2.5) node[anchor = mid] {A-A};
	\end{circuitikz}
	\end{equation*}
	\captionof{figure}{Model układu z opóźnionym wyłączaniem}
\end{figure}



%%  rys 1.15.
\begin{figure}[H]
	\begin{equation*}
	\begin{circuitikz}[american]
	\draw
	(0,0) to (6,0)
	to [V,v=$10V$] (6,3)
	(6,3) to  [european resistor, a = $R_{15}$, l= $1k\Omega$](4,3)
	(2,3) to [leD-, invert, a = LED3] (4,3)
	(2,2) node[nmos] (mos) {}
	(2,3) to (mos.drain)
	(2,0) to (mos.source)
	(mos.gate) to (0,2)
	(0,2) to [european resistor, a =$R_{16}$, l=$1M\Omega$] (0,0)
	(0,0) to (-2,0)
	to [sV,l=BNC] (-2,2)
	to (-2,2)
	(-2,0) node[ground]{}
	(-2,2) to (0,2);
	\draw [red]
	(-2,2) to (-2,6)
	(2,3) to (2,4.5)
	to (6,4.5);
	\draw
	(0,6) to (0,3)
	(0,3) node[ground] {}
	(6,5.5) to (2,5.5)
	(2,5.5) node[ground,rotate =270] {};	
	\end{circuitikz}
	\end{equation*}
\end{figure}

\section{Czas załączania tranzystora}

\section{title}
	\begin{thebibliography}{99}
		\bibitem{pa}S. Bolkowski,  \emph{Teoria obwodów elektrycznych} , ser. Elektrotechnika teoretyczna. Wydawnictwa NaukowoTechniczne,
		1986, 
		\bibitem{pa1}P. Horowitz and W. Hill, \emph{Sztuka elektroniki}. WKiŁ, 2003, vol. 1.
		\bibitem{pa2}D. Halliday, R. Resnick, and J. Walker, \emph{Podstawy fizyki.} PWN, 2003, vol. 3.
		\bibitem{pa4}J. Watson,\emph{ Elektronika.} WKiŁ, 1999.
		\bibitem{pa5}Z. Nosal and J. Baranowski, \emph{Układy elektroniczne.} WNT, 2003.
	\end{thebibliography}
	\newpage
	\tableofcontents
		
\end{document}


